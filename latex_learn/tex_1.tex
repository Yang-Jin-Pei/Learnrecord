\documentclass{article}
% ================================中文支持===============================
\usepackage{ctex} % 中文支持
% ================================颜色支持===============================
\usepackage{xcolor} % 颜色
% ================================图与表格===============================
\usepackage{graphicx} % 添加图片
\usepackage{tikz} % 绘制矢量图
\usepackage{subcaption} % 子图
\usepackage{booktabs} % 表格
\usepackage{multirow} % 多行,包括合并的表格
% ================================数学公式===============================
\usepackage{amsmath} % 数学公式
\usepackage{amssymb} % 数学符号
\usepackage{amsfonts} % 数学字体
\usepackage{bm} % 粗体
\usepackage{algorithm2e} % 算法显示
% ================================超链接===============================
\usepackage[colorlinks, linkcolor=blue, citecolor=blue, urlcolor=blue]{hyperref} % 链接为蓝色
% ================================其他===============================

\title{VSCode with LaTeX}
\author{Your Name}
\date{\today}

\begin{document}

\maketitle

%
\%

\section{一级标题}           % 用于 article
\subsection{二级标题}
\subsubsection{三级标题}
% \chapter{章}                % 用于 report 和 book


\textbf{粗体文本}
\textit{斜体文本}
\underline{下划线}

{\Large 大号字体}  % 相对大小命令:\tiny, \small, \normalsize, \large, \Large, \LARGE, \huge, \Huge

\begin{center}
这是居中的文本。
\end{center}

% 无序列表
\begin{itemize}
  \item 第一项
  \item 第二项
        \begin{itemize}
          \item 子项一
          \item 子项二
        \end{itemize}
\end{itemize}

% 有序列表
\begin{enumerate}
  \item 第一步
  \item 第二步
\end{enumerate}

行内公式:勾股定理 $a^2 + b^2 = c^2$ 非常著名。

行间公式(带编号):
\begin{equation}
E = mc^2
\label{eq:emc2} % 用于交叉引用
\end{equation}

多行对齐公式(不带编号):
\[
\begin{aligned}
f(x) &= (x+1)^2 \\
     &= x^2 + 2x + 1
\end{aligned}
\]

\begin{figure}[htbp] % h: here, t: top, b: bottom, p: page
  \centering
  \includegraphics[width=0.8\textwidth]{example-image.png} % 宽度设为文本宽度的80%
  \caption{图片的标题}
  \label{fig:my_image} % 用于交叉引用
\end{figure}

\begin{table}[htbp]
  \centering
  \caption{表格标题}
  \begin{tabular}{|l|c|r|} % l: 左对齐, c: 居中, r: 右对齐, | 表示竖线
    \hline
    姓名 & 分数 & 等级 \\ \hline %\hline表示横线
    张三 & 95 & A \\ \hline
    李四 & 87 & B+ \\ \hline
  \end{tabular}
  \label{tab:score}
\end{table}

如第\ref{sec:intro}节中图\ref{fig:arch}所示,公式(\ref{eq:main})是核心。
\%,\$,\&,\_
\textcolor{red}{这段文字是红色的}
{\color{blue} 这段文字是蓝色的}
\colorbox{yellow}{这个文本有黄色背景}
\fcolorbox{red}{white}{这个文本有红色边框和白色背景}

\cite{key}:生成一个数字标签,如 [1]。
% \citeauthor{key}:只显示作者,如 Einstein。
% \citeyear{key}:只显示年份,如 1905。
% \citet{key}:生成 "作者 (年份)" 格式(需要 `natbib` 宏包),如 Einstein (1905)。
% \citep{key}:生成 "(作者, 年份)" 格式(需要 `natbib` 宏包),如 (Einstein, 1905)。
引用多个文献:用逗号分隔,如 \cite{paper1, book1}生成

\bibliography{references}       % 使用传统 BibTeX
\bibliographystyle{plain}       % 基本数字编号
% \bibliographystyle{unsrt}       % 按引用顺序编号
% \bibliographystyle{alpha}       % 用作者缩写和年份作为标签,如 [Ein05]
% \bibliographystyle{abbrv}       % 缩写样式
% \bibliographystyle{plainnat}    % 或 apacite, etc.
\end{document}